\documentclass[a4paper]{article}
\usepackage{ctex}
\usepackage{amsmath}
\begin{document}
    \title{逻辑回归总结}
    \date{2017-9-28}
    \maketitle

    对于线性回归模型$y=\omega^T+b$输出的预测值$y$都是实值,考虑二分类任务,$y_i\in\{0,1\}$。为了将实值$y$对应到离散值,使用了 sigmoid 函数。$y=\frac{1}{1+e^{-z}}=\frac{1}{1+e^{-(\omega^T+b)}}$,对$\omega$和$b$作以下变换,令$\beta=(\omega;b)$,$\hat{x}=(x;1)$,则$y=\omega^T+b=\beta^T\hat{x}$,带入 sigmoid 函数。
	\begin{equation}\label{sigmoid}
		y=\frac{1}{1+e^{-\beta^T\hat{x}}}
	\end{equation}

	对 sigmoid 函数,$z=\ln\frac{y}{1-y}$,则式\eqref{sigmoid}可以变为
	\begin{equation}\label{probability}
		\ln\frac{y}{1-y}=\beta^T\hat{x}
	\end{equation}
	若将$y$视为类后验概率估计$P(y=1|x)$,即样本属于第一类的概率。则式\eqref{probability}可以写为:
	\begin{equation*}
		\ln\frac{P(y=1|x)}{1-P(y=1|x)}=\ln\frac{P(y=1|x)}{P(y=0|x)}=\beta^T\hat{x}
	\end{equation*}
	对于样本集$\{x_i,y_i\}^{m}_{i=1}$,样本的概率可以写为:
	\begin{equation*}
		P(y_i|\beta;\hat{x_i})=P(y_i=1|\beta;\hat{x_i})^{y_i}\left(1-P(y_i=1|\beta;\hat{x_i})\right)^{1-y_i}
	\end{equation*}
	其中:
	\begin{align}
		P(y=1|x)&=\frac{e^{\beta^Tx}}{1+e^{\beta^Tx}}\\
		P(y=0|x)&=\frac{1}{1+e^{\beta^Tx}}
	\end{align}
	n个独立的样本出现的似然函数为(因为每个样本都是独立的,所以n个样本出现的概率就是他们各自出现的概率相乘):
	\begin{equation}
	L(\theta)=\prod	P(y_i|\beta;\hat{x_i})
	\end{equation}
	对上式取对数,
	\begin{equation}\label{siran}
		\begin{split}
		L(\theta)&=\log\prod	P(y_i|\beta;\hat{x_i})\\
		&=\log \prod \left(P(y_i=1|\beta;\hat{x_i})^{y_i}\left(1-P(y_i=1|\beta;\hat{x_i})\right)^{1-y_i} \right)\\
		&=\sum\log \left(P(y_i=1|\beta;\hat{x_i})^{y_i}\left(1-P(y_i=1|\beta;\hat{x_i})\right)^{1-y_i} \right)\\
		&=\sum \left[ y_i\log P(y_i=1|\beta;\hat{x_i}) + (1-y_i)\log(1-P(y_i=1|\beta;\hat{x_i}))\right]\\
		&=\sum \left( y_i \log P(y_i=1) - y_i \log(P(y_i=0))+\log(P(y_i=0))\right)\\
		&=\sum \left( y_i \log \frac{P(y_i=1)}{P(y_i=0)} + log(\frac{1}{1+e^{\beta^T\hat{x_i}}})\right)\\
		&=\sum \left(
		y_i\beta^T\hat{x_i}-log(1+e^{\beta^T\hat{x_i}})
		\right)
		\end{split}
	\end{equation}
	为了优化求解,对目标函数取反,以进行最小值优化。

	式\eqref{siran}	取反后,对$\beta$求导数:
	\begin{equation}\label{key}
	\begin{split}
	\mathrm{d}L(\theta)
	&=\sum \left(
	\frac{e^{\beta^T\hat{x_i}} \mathrm{d}\beta^T\hat{x_i}}{1+e^{\beta^T\hat{x_i}} }
	- y_i \mathrm{d} \beta^T \hat{x_i}
	\right)\\
	&=\mathrm{tr}\sum \left(
	\frac{\hat{x_i}e^{\beta^T\hat{x_i}} \mathrm{d}\beta^T}{1+e^{\beta^T\hat{x_i}} }
	- \hat{x_i}y_i \mathrm{d} \beta^T
	\right)\\
	&=\sum \hat{x_i}(P(y_i=1|\beta;\hat{x_i})-y_i)\mathrm{d}\beta^T
	\end{split}
	\end{equation}

	\begin{equation*}
		\frac{\partial \mathrm{d}L(\theta)}{\partial \beta^T}=\sum \hat{x_i}^T(P(y_i=1|\beta;\hat{x_i})-y_i)
	\end{equation*}

	\begin{equation}
	\frac{\partial \mathrm{d}L(\theta)}{\partial \beta}=\sum \hat{x_i}(P(y_i=1|\beta;\hat{x_i})-y_i)
	\end{equation}
	运用梯度下降法进行优化求解:

	\[
			\beta_{n+1}=\beta_{n}-\alpha\frac{\partial \mathrm{d}L(\theta)}{\partial \beta}
	\]


\end{document}
